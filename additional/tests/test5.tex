%% -*- TeX-engine: luatex; ispell-language: russian -*-

\documentclass[a4paper,12pt]{article}

\usepackage[left=1.5cm,right=2cm,top=1.5cm,bottom=2cm]{geometry}

\usepackage{parskip}
\setlength{\parindent}{0mm}
\setcounter{secnumdepth}{1}

\usepackage{amsmath}

\usepackage{graphicx}

\usepackage{fontspec}
\setmainfont{PT Serif}
\newfontfamily\cyrillicfont[Script=Cyrillic,Ligatures=TeX]{PT Serif}
\setsansfont{PT Sans}
\setmonofont[Ligatures=NoCommon]{PT Mono}
\defaultfontfeatures{Ligatures=TeX}

\usepackage[bold-style=ISO]{unicode-math}
\setmathfont{XITS Math}

\usepackage{microtype}

\usepackage{hyperref}

\usepackage{polyglossia}
\setmainlanguage{russian}
\setotherlanguage{english}

\usepackage{csquotes}

%% for code snippets
\usepackage{minted}
\newminted[pycon]{pycon}{fontsize=\footnotesize}
\newminted[python3]{python3}{fontsize=\footnotesize}
\newminted[bash]{bash}{fontsize=\footnotesize}
\newmintinline[pythoninline]{python3}{fontsize=\footnotesize}
\newmintinline[bashinline]{bash}{fontsize=\footnotesize}

\pagestyle{empty}


\usepackage{blkarray}
\newcommand{\matindex}[1]{\mbox{\scriptsize#1}}

\usepackage{multicol}

\begin{document}
  \subsection*{Тест №5\hfill{19 марта 2018}}

  \makebox[\textwidth]{Представьтесь:\enspace\hrulefill}
  \paragraph{1} Напишите формулу оптимального Байесовского алгоритма классификации.
  
  \makebox[\linewidth]{\hrulefill}
  \makebox[\linewidth]{\hrulefill}
  
  \paragraph{2} Векторизуйте с использованием словаря
  $$
  V = \{\text{was}, \text{agree}, \text{tea}, \text{think},
        \text{I}, \text{it}, \text{you}, \text{do}, \text{about}\}
  $$
  следующие предложения:
  \begin{itemize}
  \item Molly, my sister and I fell out,
  \item And what do you think it was all about?
  \item She loved coffee and I loved tea,
  \item And that was the reason we couldn't agree.
  \end{itemize}

  \paragraph{3} Какие две подзадачи необходимо решить для обучения Байесовского
  классификатора?

  \makebox[\linewidth]{\hrulefill}
  \makebox[\linewidth]{\hrulefill}
  \makebox[\linewidth]{\hrulefill}
  \makebox[\linewidth]{\hrulefill}

  \paragraph{4} Какой класс будет назначен примеру $x = [1, 0, 1]$
  наивным Байесовским классификатором с параметрами
  \begin{center}
    \begin{tabular}{l|c|ccc}
      $y \in Y$   & $\hat{P}_y$ & $\hat{\theta}_{y0}$ & $\hat{\theta}_{y1}$ & $\hat{\theta}_{y2}$ \\
      \hline
      ${-}1$      & 0.5    & 0.3      & 0.1     & 0.6   \\
      ${+}1$      & 0.5    & 0.5      & 0.1     & 0.4
    \end{tabular}
  \end{center}

  \vspace{4em}
  
  \paragraph{5} Какую проблему решает предположение \emph{наивности} в наивном
  Байесовском классификаторе?

  \makebox[\linewidth]{\hrulefill}
  \makebox[\linewidth]{\hrulefill}
  \makebox[\linewidth]{\hrulefill}
  \makebox[\linewidth]{\hrulefill}
  
\end{document}
