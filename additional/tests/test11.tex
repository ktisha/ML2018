%% -*- TeX-engine: luatex; ispell-language: russian -*-

\documentclass[a4paper,12pt]{article}

\input{../handout-base}

\usepackage{blkarray}
\newcommand{\matindex}[1]{\mbox{\scriptsize#1}}

\usepackage{multicol}

\begin{document}
  \subsection*{Тест №11\hfill{14 мая 2018}}

  \makebox[\textwidth]{Представьтесь:\enspace\hrulefill}
  
  \paragraph{1} Как можно по кривой обучения понять, что выбрана слишком сложная модель и она недообучается?

  \makebox[\linewidth]{\hrulefill}
  \makebox[\linewidth]{\hrulefill}

  \paragraph{2} Чем по смыслу является bias? Variance?
	
  \makebox[\linewidth]{\hrulefill}
  \makebox[\linewidth]{\hrulefill}
  \makebox[\linewidth]{\hrulefill}
	
  \paragraph{3} Как подбирается параметр регуляризации в гребневой регрессии?

  \makebox[\linewidth]{\hrulefill}
  \makebox[\linewidth]{\hrulefill}
  \makebox[\linewidth]{\hrulefill}
  \makebox[\linewidth]{\hrulefill}

  \paragraph{4} Какое требование накладывается на новые признаки в методе главных компонент?

  \makebox[\linewidth]{\hrulefill}
  \makebox[\linewidth]{\hrulefill}
  \makebox[\linewidth]{\hrulefill}
  \makebox[\linewidth]{\hrulefill}

  \paragraph{5} Дайте определение среднего метода (гипотезы).

  \makebox[\linewidth]{\hrulefill}
  \makebox[\linewidth]{\hrulefill}
  \makebox[\linewidth]{\hrulefill}
  \makebox[\linewidth]{\hrulefill}
  \makebox[\linewidth]{\hrulefill}
 
\end{document}
